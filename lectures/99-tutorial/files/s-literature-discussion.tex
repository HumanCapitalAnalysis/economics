 %-------------------------------------------------------------------------------
\section{Literature discussion}
%-------------------------------------------------------------------------------
Please consider the research articles by \citet{Lagakos.2018} and \citet{Carneiro.2011} we discussed in class.  Your task going forward is to explore the commonalities and differences between them concerning research questions, core findings, theoretical foundations, empirical challenges,and estimation strategies. \\

Both articles represent seminal work in their respective field.  However, we also discussed \citet{Lagakos.2018} as a shining example of clear, concise, and well structured academic writing.  Please adhere to the same principles when crafting your answers below.  Ample time is allocated to each question.  The quality of your writing is taken into account for grading purposes.
\begin{boenumerate}
\item  Please describe the research question and core findings using five sentences or less for each of the two articles.
\end{boenumerate}\vspace{0.3cm}

The Mincer equation \citep{Mincer.1974} is featured prominently in both articles.  However, each ofthem emphasizes different aspects of it.  \citet{Lagakos.2018} focus on the return to experience,while \citet{Carneiro.2011} concentrate on the return to schooling.

\begin{boenumerate}
\item Briefly  describe  the  two  economic  models  that  provide  the  theoretical  foundation  for  the Mincer equation.  What are their common features?  How do they differ?  Please keep the useof mathematical notation to a minimum.
\item Please provide and discuss some examples for different sources of heterogeneity in returns toexperience and schooling within and across countries?
\item How and to what degree does the estimation strategy in both papers account for them?
\end{boenumerate}\vspace{0.3cm}
