%-------------------------------------------------------------------------------
\FloatBarrier\section{Quiz}
%-------------------------------------------------------------------------------
Please mark each claim as either true or false. Make sure to outline your reasoning briefly.

\begin{boenumerate}
%-------------------------------------------------------------------------------
%-------------------------------------------------------------------------------
\item Consider the seminal model by \cite{Spence.1973} as presented in class. High productivity workers always prefer the option to signal their ability.

\item Based on the discussion in class, the option value of schooling is always strictly positive for all schooling transitions.

\item The accounting-identity model as presented in \cite{Heckman.2006a} provides a justification for interpreting the Mincer coefficient as an internal rate of return.

\item A college tuition subsidy always leads to an increase in high school graduation rates regardless of the individual time preferences.

\item \cite{Heckman.2006b} report that the effect of cognitive skills on social outcomes is always more pronounced than the effect of noncognitive skills.

\item \cite{Keane.1997} find that a basic model of human capital investment explains the observed investment patterns just as well as their extended model.

\item \cite{Carneiro.2011} find that individuals make their schooling decisions in light of heterogeneous returns.

\item \cite{Keane.1997} point to heterogeneous schooling levels at age 16 as the main determinant of inequality in expected total lifetime utility.

\item \cite{Heckman.2006a} compile several pieces of empirical evidence that point towards a rejection of the standard Mincer regression model.

\item \cite{Lagakos.2018} find that wages increase substantially more over the life cycle in poor countries than in rich countries.

\item \cite{Lagakos.2018} show that their core findings hold up regardless of whether they focus on part or full time male wage workers.

\item \cite{Lagakos.2018} determine that differences in long-term contracts are an important driver of cross-country differences in life cycle wage growth.

\item \cite{Lagakos.2018} determine that human capital or search frictions are promising explanations for the cross-country differences in life cycle wage growth.

\item \cite{Keane.1997} find that the predictions of life cycle choices from a static and dynamic model of human capital investments are in general agreement.

\item \cite{Keane.1997} find that a \$2,000 college tuition subsidy has a pronounced impact on the expected present value of lifetime utility.

\item \cite{Carneiro.2011} report point estimates that the marginal benefit of treatment for the average individual remains positive when moving along the distribution of $V$.

\item \cite{Heckman.2006a} restrict their analysis to the ex post return to schooling.
%-------------------------------------------------------------------------------
%-------------------------------------------------------------------------------
\end{boenumerate}
