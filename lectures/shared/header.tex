%!TEX root = ../main.tex

\RequirePackage{bibentry}
\makeatletter\let\saved@bibitem\@bibitem\makeatother

\documentclass{beamer}
\makeatletter\let\@bibitem\saved@bibitem\makeatother

\renewcommand{\baselinestretch}{1.2}\normalsize
\usetheme{default}
\setbeamertemplate{navigation symbols}{}
\setbeamertemplate{footline}[frame number]

\usepackage{verbatim}
\usepackage{etex}

% BIBLIOGRAPHY APACITE
\usepackage[apaciteclassic]{apacite}
\usepackage{notoccite}
\usepackage{bibentry}
\usepackage{pdfpages}
\usepackage{natbib}

% MATHEMATICS AND FONTS
\DeclareMathOperator*{\argmin}{arg\,min}
\DeclareMathOperator*{\argmax}{arg\,max}

\renewcommand{\vec}[1]{\mathbf{#1}}

\usepackage{amsfonts}
\usepackage{amsmath}
\usepackage{amssymb}
\usepackage{bbm}
\usepackage{algpseudocode}
\usepackage{setspace}

\usepackage{arev}
\usepackage[latin1]{inputenc}
\usepackage[T1]{fontenc}


% GRAPHS
\usepackage{graphicx}
\usepackage{subfig}
\usepackage{caption}
\graphicspath{{material/}}
\usepackage{relsize}
\usepackage{lscape}
\usepackage{fancybox}
\usepackage{epstopdf}


% TABLES AND OTHER ENVIRONMENTS
\usepackage{tabularx}
\usepackage{longtable}
\usepackage{booktabs}
\usepackage{color,colortbl}
\usepackage{threeparttable}

\usepackage{enumerate}

\usepackage{fix-cm}

\usepackage{bookmark}
\usepackage{hyperref}
\hypersetup{colorlinks=true,urlcolor=blue,citecolor=black}


\usepackage{tikz}
\tikzset{
	treenode/.style = {shape=rectangle, rounded corners,
		draw, align=center,
		top color=white, bottom color=blue!20},
	root/.style     = {treenode, font=\Large, bottom color=red!30},
	env/.style      = {treenode, font=\ttfamily\normalsize},
	dummy/.style    = {circle,draw}
}
%\usepackage{cmbright}
\def\newblock{\hskip .11em plus .33em minus .07em}
\newcommand{\bs}{\boldsymbol}
\newcommand{\N}{\mathbb{N}}
\newcommand{\cov}{\mathrm{cov}\thin}
\newcommand{\thin}{\thinspace}
\newcommand{\thick}{\thickspace}
\newcommand{\Lim}[1]{\raisebox{0.5ex}{\scalebox{0.8}{$\displaystyle \lim_{#1}\;$}}}

\newcommand{\vect}[1]{\mathbf{#1}}
\newcommand{\myfrac}[3][0pt]{\genfrac{}{}{}{}{\raisebox{#1}{$#2$}}{\raisebox{-#1}{$#3$}}}
\newcommand{\U}{\mathrm{U}}	%Uniform Distribution
\newcommand{\D}{\mathrm{D}}	%Dirichlet Distribution
\newcommand{\W}{\mathrm{W}}	%Wishart Distribution
\newcommand{\E}{\mathrm{E}}		%Expectation
\newcommand{\Prob}{\mbox{Pr}}		%Expectation
\newcommand{\Iden}{\mathbb{I}}	%Identity Matrix
\newcommand{\Ind}{\mathrm{I}}	%Indicator Function
\newcommand{\Tau}{\mathcal{T}\thin}

\newcommand{\var}{\mathrm{var}\thin}
\newcommand{\plim}{\mathrm{plim}\thin}
\newcommand\indep{\protect\mathpalette{\protect\independenT}{\perp}}
\def\independenT#1#2{\mathrel{\rlap{$#1#2$}\mkern5mu{#1#2}}}
\newcommand{\notindep}{\ensuremath{\perp\!\!\!\!\!\!\diagup\!\!\!\!\!\!\perp}}%

\newcommand{\mc}{\multicolumn}

\newcommand{\ph}{\phantom}
% weitere Optionen:
% secbar: Gliederung im Kopf, nur sections (alternativ zu subsecbar)
% handout: Produktion von Handouts, keine Animationen
\definecolor{darkblue}{rgb}{0,.35,.62}
\definecolor{lightblue}{rgb}{0.8,0.85,1}
\definecolor{lightgrey}{gray}{0.1}	%Farben mischen

%	kbordermatrix options

\makeatletter
\newcommand{\vast}{\bBigg@{4}}
\newcommand{\Vast}{\bBigg@{5}}
\makeatother
\newcommand{\indicator}[1]{\mathbbm{1}{\left\{ {#1} \right\} }}
\newcommand{\indic}{1{\hskip -2.5 pt}\hbox{1} }


\definecolor{lightgrey}{gray}{0.90}	%Farben mischen
\definecolor{grey}{gray}{0.85}
\definecolor{darkgrey}{gray}{0.65}
\definecolor{lightblue}{rgb}{0.8,0.85,1}

\renewcommand{\arraystretch}{1.5}


\usepackage{tikz}
\usetikzlibrary{trees,shapes,arrows,decorations.pathmorphing,backgrounds,positioning,fit,petri}
\renewcommand*{\familydefault}{\sfdefault}

\tikzset{forestyle/.style = {rectangle, thick, minimum width = 5cm, minimum height = 0.5cm, text width = 4.5cm, outer sep = 1mm},
	pre/.style={<-, shorten <=1pt, >=stealth, ultra thick},
	extend/.style={<-,dashed, shorten <=1pt, >=stealth, ultra thick}}
\captionsetup[subfigure]{labelformat=empty}


\newcommand{\beginbackup}{
	\newcounter{framenumbervorappendix}
	\setcounter{framenumbervorappendix}{\value{framenumber}}
}
\newcommand{\backupend}{
	\addtocounter{framenumbervorappendix}{-\value{framenumber}}
	\addtocounter{framenumber}{\value{framenumbervorappendix}}
}


% Begin Full Justification ---------------------------------------------------------

\usepackage{ragged2e}
% \usepackage{etoolbox}
\usepackage{lipsum}
\makeatletter
\renewcommand{\itemize}[1][]{%
	\beamer@ifempty{#1}{}{\def\beamer@defaultospec{#1}}%
	\ifnum \@itemdepth >2\relax\@toodeep\else
	\advance\@itemdepth\@ne
	\beamer@computepref\@itemdepth% sets \beameritemnestingprefix
	\usebeamerfont{itemize/enumerate \beameritemnestingprefix body}%
	\usebeamercolor[fg]{itemize/enumerate \beameritemnestingprefix body}%
	\usebeamertemplate{itemize/enumerate \beameritemnestingprefix body begin}%
	\list
	{\usebeamertemplate{itemize \beameritemnestingprefix item}}
	{\def\makelabel##1{%
			{%
				\hss\llap{{%
						\usebeamerfont*{itemize \beameritemnestingprefix item}%
						\usebeamercolor[fg]{itemize \beameritemnestingprefix item}##1}}%
			}%
		}%
	}
	\fi%
	\beamer@cramped%
	\justifying% NEW
	%\raggedright% ORIGINAL
	\beamer@firstlineitemizeunskip%
}

\justifying

% \apptocmd{\frame}{\justifying}{}{}

\usepackage{array}
\newcolumntype{L}[1]{>{\raggedright\let\newline\\\arraybackslash\hspace{0pt}}m{#1}}
\newcolumntype{C}[1]{>{\centering\let\newline\\\arraybackslash\hspace{0pt}}m{#1}}
\newcolumntype{R}[1]{>{\raggedleft\let\newline\\\arraybackslash\hspace{0pt}}m{#1}}



% End Full Justification ------------------------------------------------------------
