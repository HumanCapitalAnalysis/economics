\begin{boenumerate}
%-------------------------------------------------------------------------------
%-------------------------------------------------------------------------------
\item Consider the seminal model by \cite{Spence.1973} as presented in class. High productivity workers always prefer the option to signal their ability.

\item Based on the discussion in class, the option value of schooling is always strictly positive for all schooling transitions.

\item The accounting-identity model as presented in \cite{Heckman.2006a} provides a justification for interpreting the Mincer coefficient as an internal rate of return.

\item A college tuition subsidy always leads to an increase in high school graduation rates regardless of the individual time preferences.

\item \cite{Heckman.2006b} report that the effect of cognitive skills on social outcomes is always more pronounced than the effect of noncognitive skills.

\item \cite{Keane.1997} find that a basic model of human capital investment explains the observed investment patterns just as well as their extended model.

\item \cite{Carneiro.2011} find that individuals make their schooling decisions in light of heterogeneous returns.

\item \cite{Keane.1997} point to heterogeneous schooling levels at age 16 as the main determinant of inequality in expected total lifetime utility.

\item \cite{Heckman.2006a} compile several pieces of evidence that point towards a rejection of the standard Mincer regression model.
%-------------------------------------------------------------------------------
%-------------------------------------------------------------------------------
\end{boenumerate}
